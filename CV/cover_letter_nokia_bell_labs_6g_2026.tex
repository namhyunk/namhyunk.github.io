% Cover Letter — Nokia Bell Labs (Murray Hill, NJ)
\documentclass[11pt]{article}

\usepackage[utf8]{inputenc}
\usepackage[letterpaper,margin=1in]{geometry}
\usepackage[hidelinks]{hyperref}
\usepackage{parskip}

\begin{document}

{\Large \textbf{Namhyun Kim}}\\
Tempe, AZ, USA\\
Email: \href{mailto:namhyun@asu.edu}{namhyun@asu.edu}\\
Website: \href{https://namhyunk.github.io/}{https://namhyunk.github.io/}

\vspace{0.75em}
January 14, 2026

\vspace{0.75em}
Hiring Committee\\
Nokia Bell Labs\\
Murray Hill, New Jersey (On-site)

\vspace{0.75em}
Dear Hiring Committee,

I am writing to apply for a Summer 2026 research internship at Nokia Bell Labs in Murray Hill (June 1 -- August 7, 2026). I am a Ph.D. student in Electrical, Computer and Energy Engineering at Arizona State University, advised by Prof. Ahmed Alkhateeb. My work combines wireless communications and signal processing with modern machine learning, and I am excited about the opportunity to contribute to high-impact 6G research spanning physical-layer algorithms, feedback, (de)coding, and data-driven wireless system design.

My research background fits the role's emphasis on fundamentals, benchmarking, and publication-quality outcomes. I have published in \textit{IEEE Transactions on Wireless Communications} (2025) and presented at \textit{IEEE ISIT 2025}, with projects that study downlink MIMO and integrated sensing and communications (ISAC) under practical constraints such as limited feedback and imperfect channel knowledge. Across these projects, I focus on clearly defining evaluation criteria, comparing against strong baselines, and documenting insights in a form that can translate into technical reports and, when appropriate, conference/journal submissions.

I also recently released \textit{LWM-Spectro} (\href{https://arxiv.org/abs/2601.08780}{arXiv} / \href{https://huggingface.co/spaces/wi-lab/LWM-Spectro}{Hugging Face}), a transformer-based foundation model pretrained on large-scale received baseband I/Q spectrograms. The work combines self-supervised masked modeling, contrastive learning, and a mixture-of-experts (MoE) architecture to learn transferable wireless representations that perform strongly on downstream tasks even with minimal labeled data. This project reflects hands-on experience in end-to-end research execution: data generation/curation, model design and tuning, rigorous evaluation, and clear technical write-up.

In addition, prior to starting my Ph.D., I worked as a manager in SK Telecom's LTE/5G RAN performance improvement group, where I contributed to L1/L2/L3 planning, troubleshooting, and performance analysis, and participated in commercial verification testing of 64TRX massive MIMO systems. This experience strengthened my ability to troubleshoot complex wireless systems and work effectively in collaborative teams.

I would welcome the chance to contribute to Bell Labs projects by (1) identifying state-of-the-art approaches, (2) proposing and analyzing new solutions, (3) benchmarking against legacy baselines, and (4) delivering clear reports and presentations. Thank you for your time and consideration.

Sincerely,\\
Namhyun Kim

\end{document}
