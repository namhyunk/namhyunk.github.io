% Cover Letter — Nokia Bell Labs (Network Architecture Research: Digital Twins + AI for RAN)
\documentclass[11pt]{article}

\usepackage[utf8]{inputenc}
\usepackage[letterpaper,margin=1in]{geometry}
\usepackage[hidelinks]{hyperref}
\usepackage{parskip}

\begin{document}

{\Large \textbf{Namhyun Kim}}\\
Tempe, AZ, USA\\
Email: \href{mailto:namhyun@asu.edu}{namhyun@asu.edu}\\
Website: \href{https://namhyunk.github.io/}{https://namhyunk.github.io/}

\vspace{0.75em}
January 14, 2026

\vspace{0.75em}
Hiring Committee\\
Nokia Bell Labs -- Network Architecture Research\\
Murray Hill, NJ (On-site)

\vspace{0.75em}
Dear Hiring Committee,

I am writing to apply for the Bell Labs internship role focused on \textit{Digital Twins and AI for RAN} (Jun--Aug 2026, on-site in Murray Hill, NJ). I am a Ph.D. student in Electrical, Computer and Energy Engineering at Arizona State University, advised by Prof. Ahmed Alkhateeb. My work combines wireless communications fundamentals with modern machine learning, and I enjoy building software prototypes and clear demos that make research outcomes tangible.

This position is exciting to me because it bridges RAN understanding with software systems: defining and implementing solutions, integrating open-source and cloud components, and demonstrating end-to-end capabilities. I have strong Python experience and hands-on ML practice (model design, tuning, evaluation, and debugging), along with a solid foundation in communication theory and wireless protocols.

In my research, I develop and evaluate wireless algorithms under realistic constraints such as limited feedback, imperfect channel knowledge, and diverse channel/hardware impairments, with publications in \textit{IEEE Transactions on Wireless Communications} (2025) and a presentation at \textit{IEEE ISIT 2025}. I also recently released \textit{LWM-Spectro} (\href{https://arxiv.org/abs/2601.08780}{arXiv} / \href{https://huggingface.co/spaces/wi-lab/LWM-Spectro}{Hugging Face}), where I built a pipeline to generate and curate large-scale \emph{received} baseband I/Q signals and trained a transformer-based model using self-supervised learning and a mixture-of-experts (MoE) architecture. The project reflects how I approach applied research: define metrics, establish baselines, build reproducible software, and communicate results clearly.

Before my Ph.D., I worked at SK Telecom in the LTE/5G RAN performance improvement group, contributing to L1/L2/L3 planning, troubleshooting, and performance analysis, and participating in commercial verification testing of 64TRX massive MIMO systems. This experience gave me practical context on PHY/MAC and higher-layer behaviors in operational networks, and trained me to derive actionable insights from measurements and logs.

I would welcome the opportunity to contribute to Bell Labs by implementing software for RAN digital twins, integrating simulators and ML components, and delivering demos and technical documentation for internal presentations and reports. Thank you for your time and consideration.

Sincerely,\\
Namhyun Kim

\end{document}
