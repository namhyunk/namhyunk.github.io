% Cover Letter — Samsung Research America (SMI Lab)
\documentclass[11pt]{article}

\usepackage[utf8]{inputenc}
\usepackage[letterpaper,margin=1in]{geometry}
\usepackage[hidelinks]{hyperref}
\usepackage{parskip}

\begin{document}

{\Large \textbf{Namhyun Kim}}\\
Tempe, AZ, USA\\
Email: \href{mailto:namhyun@asu.edu}{namhyun@asu.edu}\\
Website: \href{https://namhyunk.github.io/}{https://namhyunk.github.io/}

\vspace{0.75em}
January 14, 2026

\vspace{0.75em}
Hiring Committee\\
Samsung Research America -- Standards \& Mobility Innovation (SMI) Lab\\
6105 Tennyson Pkwy, Suite 300\\
Plano, TX 75024

\vspace{0.75em}
Dear Hiring Committee,

I am excited to apply for the \textit{2026 Intern, 6G Cellular Standardization (Spring/Summer)} position at Samsung Research America. I am a Ph.D. student in Electrical, Computer and Energy Engineering at Arizona State University, advised by Prof. Ahmed Alkhateeb. My work focuses on beyond-5G/6G wireless systems, massive MIMO, and learning-based methods for wireless. With hands-on LTE/5G RAN experience from SK Telecom and recent research contributions, I am eager to support Samsung's 6G-Ready (6GR) and 3GPP-relevant standardization efforts.

At SK Telecom, I worked in the LTE/5G RAN performance improvement group, contributing to L1/L2/L3 planning, troubleshooting, and performance analysis. I also participated in commercial verification testing of 64TRX massive MIMO systems in collaboration with Samsung Electronics, which strengthened my ability to communicate tradeoffs clearly and work effectively across teams.

In my Ph.D. research, I study next-generation wireless systems, including downlink MIMO and integrated sensing and communications (ISAC). This work led to publications in \textit{IEEE Transactions on Wireless Communications} (2025) and a presentation at \textit{IEEE ISIT 2025}. I regularly work under practical constraints such as limited feedback and imperfect channel knowledge, which aligns well with robust and efficient access/control signaling design.

Most recently, I released \textit{LWM-Spectro} (\href{https://arxiv.org/abs/2601.08780}{arXiv} / \href{https://huggingface.co/spaces/wi-lab/LWM-Spectro}{Hugging Face}), where I built a standards-aware pipeline to generate and curate large-scale \emph{received} baseband I/Q signals (represented as spectrograms) across diverse waveforms, channels, and realistic impairments. On top of this data, I pretrained a transformer-based foundation model using self-supervised masked modeling, contrastive learning, and a mixture-of-experts (MoE) design, achieving strong transfer to downstream tasks even with minimal labeled data. This experience closely matches standardization-oriented work: turning specifications into reproducible evaluations and using data-driven tools to inform standard-aligned design choices under practical constraints.

I am confident my background matches the role's requirements, including strong fundamentals in cellular communications and hands-on proficiency with MATLAB, C/C++, and Python-based analysis workflows. I would be thrilled to contribute to research that leads to impactful 3GPP-relevant contributions and potential IPR development within the SMI Lab.

Thank you for your time and consideration. I would welcome the opportunity to discuss how my background aligns with Samsung Research America's 6G standardization efforts.

Sincerely,\\
Namhyun Kim

\end{document}
