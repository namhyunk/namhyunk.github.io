% Cover Letter — Samsung Research America (SMI Lab)
\documentclass[11pt]{article}

\usepackage[utf8]{inputenc}
\usepackage[letterpaper,margin=1in]{geometry}
\usepackage[hidelinks]{hyperref}
\usepackage{parskip}

\begin{document}

{\Large \textbf{Namhyun Kim}}\\
Tempe, AZ, USA\\
Email: \href{mailto:namhyun@asu.edu}{namhyun@asu.edu}\\
Website: \href{https://namhyunk.github.io/}{https://namhyunk.github.io/}

\vspace{0.75em}
January 14, 2026

\vspace{0.75em}
Hiring Committee\\
Samsung Research America -- Standards \& Mobility Innovation (SMI) Lab\\
6105 Tennyson Pkwy, Suite 300\\
Plano, TX 75024

\vspace{0.75em}
Dear Hiring Committee,

I am writing to apply for the \textit{2026 Intern, 5G Terminal Performance Enhancement Research (Summer)} position at Samsung Research America (SMI). I am a Ph.D. student in Electrical, Computer and Energy Engineering at Arizona State University, advised by Prof. Ahmed Alkhateeb. My background combines hands-on LTE/5G RAN engineering with research in next-generation wireless systems and learning-based methods, and I am excited about the opportunity to work on data-driven approaches to improve real-world UE performance.

At SK Telecom, I worked in the LTE/5G RAN performance improvement group, where I gained practical experience in L1/L2/L3 planning, troubleshooting, and performance analysis using large-scale network logs and KPI-driven workflows. I also participated in commercial verification testing of 64TRX massive MIMO systems in collaboration with Samsung Electronics. This experience trained me to design measurement scenarios, diagnose root causes from logs, and communicate actionable recommendations---skills directly aligned with terminal performance enhancement research.

In my Ph.D. research, I develop wireless algorithms under practical constraints such as limited feedback, imperfect channel knowledge, and realistic channel/hardware impairments, with publications in \textit{IEEE Transactions on Wireless Communications} (2025) and a presentation at \textit{IEEE ISIT 2025}. I routinely use simulation-driven tradeoff studies (MATLAB/Python) and reproducible evaluation to compare alternatives and translate conceptual ideas into measurable requirements.

I also recently released \textit{LWM-Spectro} (\href{https://arxiv.org/abs/2601.08780}{arXiv} / \href{https://huggingface.co/spaces/wi-lab/LWM-Spectro}{Hugging Face}), where I built a standards-aware pipeline to generate and curate large-scale \emph{received} baseband I/Q signals and trained a transformer-based foundation model with self-supervised learning and a mixture-of-experts (MoE) design. Beyond the model itself, the project reflects how I approach data collection, experiment design, and analysis---including building clean datasets, using Python tooling (e.g., Pandas), and evaluating performance across diverse conditions.

I would be excited to contribute to SMI by (1) designing data collection plans and scenarios, (2) analyzing device and network logs to validate hypotheses, and (3) developing and verifying algorithms via simulations and prototyping with clear success criteria. Thank you for your time and consideration. I would welcome the opportunity to discuss how my background fits SMI's terminal performance enhancement work.

Sincerely,\\
Namhyun Kim

\end{document}
