% Cover Letter — Samsung Research America (SMI Lab)
\documentclass[11pt]{article}

\usepackage[utf8]{inputenc}
\usepackage[letterpaper,margin=1in]{geometry}
\usepackage[hidelinks]{hyperref}
\usepackage{parskip}

\begin{document}

{\Large \textbf{Namhyun Kim}}\\
Tempe, AZ, USA\\
Email: \href{mailto:namhyun@asu.edu}{namhyun@asu.edu}\\
Website: \href{https://namhyunk.github.io/}{https://namhyunk.github.io/}

\vspace{0.75em}
January 14, 2026

\vspace{0.75em}
Hiring Committee\\
Samsung Research America -- Standards \& Mobility Innovation (SMI) Lab\\
6105 Tennyson Pkwy, Suite 300\\
Plano, TX 75024

\vspace{0.75em}
Dear Hiring Committee,

I am writing to apply for the \textit{2026 Research Intern, AI Algorithm Design for 6G (Summer)} position at Samsung Research America (SMI). I am a Ph.D. student in Electrical, Computer and Energy Engineering at Arizona State University, advised by Prof. Ahmed Alkhateeb. My research sits at the intersection of wireless communications and machine learning, with a focus on practical 5G/6G system design, MIMO/OFDM-based air interfaces, and standards-relevant evaluation.

I am excited about this role because it combines (i) 6G wireless AI algorithm design and simulation, (ii) interface design considerations aligned with 3GPP and O-RAN directions, and (iii) strong emphasis on technical documentation and IPR. In my prior industry role at SK Telecom (LTE/5G RAN performance improvement group), I worked on L1/L2/L3 planning, troubleshooting, and performance analysis, and participated in commercial verification testing of 64TRX massive MIMO systems in collaboration with Samsung Electronics. This experience helped me translate engineering constraints into clear evaluation criteria and communicate results effectively across cross-functional teams.

In my Ph.D. research, I develop wireless algorithms under realistic constraints such as limited feedback, imperfect channel knowledge, and hardware/channel impairments. This work has led to publications in \textit{IEEE Transactions on Wireless Communications} (2025) and a presentation at \textit{IEEE ISIT 2025}. Across projects, I rely on simulation-driven tradeoff studies (MATLAB/Python) and reproducible evaluation workflows to compare candidate techniques and motivate practical design choices.

More recently, I released \textit{LWM-Spectro} (\href{https://arxiv.org/abs/2601.08780}{arXiv} / \href{https://huggingface.co/spaces/wi-lab/LWM-Spectro}{Hugging Face}), where I built a standards-aware pipeline to generate and curate large-scale \emph{received} baseband I/Q signals (represented as spectrograms) across diverse waveforms, channels, and realistic impairments. On top of this data, I pretrained a transformer-based foundation model using self-supervised masked modeling, contrastive learning, and a mixture-of-experts (MoE) design. This project reflects my ability to (1) define measurable requirements, (2) design simulation and evaluation protocols, and (3) document results clearly---skills that are directly relevant to 6G wireless AI algorithm development and proof-of-concept design.

I would be thrilled to contribute to SMI's research on wireless standards and fast prototyping use cases, and to support technical reporting and IPR generation. Thank you for your time and consideration, and I would welcome the opportunity to discuss how my background fits the SMI team's goals.

Sincerely,\\
Namhyun Kim

\end{document}
