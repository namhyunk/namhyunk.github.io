% Cover Letter — Samsung Research America (SMI Lab)
\documentclass[11pt]{article}

\usepackage[utf8]{inputenc}
\usepackage[letterpaper,margin=1in]{geometry}
\usepackage[hidelinks]{hyperref}
\usepackage{parskip}

\begin{document}

{\Large \textbf{Namhyun Kim}}\\
Tempe, AZ, USA\\
Email: \href{mailto:namhyun@asu.edu}{namhyun@asu.edu}\\
Website: \href{https://namhyunk.github.io/}{https://namhyunk.github.io/}

\vspace{0.75em}
January 14, 2026

\vspace{0.75em}
Hiring Committee\\
Samsung Research America -- Standards \& Mobility Innovation (SMI) Lab\\
6105 Tennyson Pkwy, Suite 300\\
Plano, TX 75024

\vspace{0.75em}
Dear Hiring Committee,

I am writing to apply for the \textit{2026 Intern, AI Wireless Research (Spring/Summer)} position at Samsung Research America (SMI). I am a Ph.D. student in Electrical, Computer and Energy Engineering at Arizona State University, advised by Prof. Ahmed Alkhateeb. My research focuses on practical wireless system design and on applying modern deep learning to wireless signals and networks, with an emphasis on standards-relevant evaluation.

This position is an excellent fit for my background because it combines wireless system research with end-to-end AI workflows: designing models, tuning and debugging them carefully, integrating them with wireless simulators, and turning high-level concepts into clear specifications and measurable performance gains.

In my research, I develop wireless algorithms under realistic constraints (limited feedback, imperfect channel knowledge, and hardware/channel impairments), with publications in \textit{IEEE Transactions on Wireless Communications} (2025) and a presentation at \textit{IEEE ISIT 2025}. I routinely build simulation-driven tradeoff studies in MATLAB/Python and maintain reproducible evaluation pipelines to compare alternatives and document results clearly.

More recently, I released \textit{LWM-Spectro} (\href{https://arxiv.org/abs/2601.08780}{arXiv} / \href{https://huggingface.co/spaces/wi-lab/LWM-Spectro}{Hugging Face}), where I built a standards-aware pipeline to generate and curate large-scale \emph{received} baseband I/Q signals (represented as spectrograms). On top of this data, I pretrained a transformer-based foundation model using self-supervised masked modeling, contrastive learning, and a mixture-of-experts (MoE) architecture. This project reflects hands-on experience with deep learning architecture design, hyper-parameter tuning, evaluation across diverse conditions, and performance analysis/debugging using Python and common ML tooling.

I also bring industry experience from SK Telecom's LTE/5G RAN performance improvement group, where I worked on L1/L2/L3 planning, troubleshooting, and performance analysis and participated in commercial verification testing of 64TRX massive MIMO systems in collaboration with Samsung Electronics. That experience trained me to define evaluation criteria, interpret logs and measurements, and communicate results effectively across teams.

I would be excited to contribute to SMI by developing AI-assisted wireless solutions across PHY and higher layers, integrating wireless simulation with ML environments for AI-in-the-loop evaluation, and producing clear technical documentation that supports standards-aligned design decisions. Thank you for your time and consideration. I would welcome the opportunity to discuss how my background fits the SMI team's goals.

Sincerely,\\
Namhyun Kim

\end{document}
